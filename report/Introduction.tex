\chapter{Introduction To Database Management System}

\section{Introduction}
Databases and database technology have a major impact on the growing use of computers. It is fair to say that databases play a critical role in almost all areas where computers are used, including business, electronic commerce, engineering, medicine, genetics, law, education, and library science. The word database is so commonly used that we must begin by defining what a database is. Our initial definition is quite general. A database is a collection of related data. By data, we mean known facts that can be recorded and that have implicit meaning. For example, consider the names, telephone numbers, and addresses of the people you know. You may have recorded this data in an indexed address book or you may have stored it on a hard drive, using a personal computer and software such as Microsoft Access or Excel. This collection of related data with an implicit meaning is a database. The preceding definition of database is quite general; for example, we may consider the collection of words that make up this page of text to be related data and hence to constitute a database. However, the common use of the term database is usually more restricted. 
A database has the following implicit properties:

\begin{itemize}
\item{A database represents some aspect of the real world, sometimes called the miniworld or the universe of discourse (UoD). Changes to the miniworld are reflected in the database.}
\item{A database is a logically coherent collection of data with some inherent meaning. A random assortment of data cannot correctly be referred to as a database}
\item{A database is designed, built, and populated with data for a specific purpose. It has an intended group of users and some preconceived applications in which these users are interested}
\end{itemize}
A database management system (DBMS) is a collection of programs that enables users to create and maintain a database. The DBMS is a general-purpose software system that facilitates the processes of defining, constructing, manipulating, and sharing databases among various users and applications. Defining a database involves specifying the data types, structures, and constraints of the data to be stored in the database. The database definition or descriptive information is also stored by the DBMS in the form of a database catalog or dictionary; it is called meta-data. Constructing the database is the process of storing the data on some storage medium that is controlled by the DBMS. Manipulating a database includes functions such as querying the database to retrieve specific data, updating the database to reflect changes in the miniworld, and generating reports from the data. Sharing a database allows multiple users and programs to access the database simultaneously.

\thispagestyle{fancy}

\section{History of DBMS }
In 1959, the TX-2 computer was developed at MIT's Lincoln Laboratory. The TX-2 integrated a number of new man-machine interfaces. A light pen could be used to draw sketches on the computer using Ivan Sutherland's revolutionary Sketchpad software.[4] Using a light pen, Sketchpad allowed one to draw simple shapes on the computer screen, save them and even recall them later. The light pen itself had a small photoelectric cell in its tip. This cell emitted an electronic pulse whenever it was placed in front of a computer screen and the screen's electron gun fired directly at it. By simply timing the electronic pulse with the current location of the electron gun, it was easy to pinpoint exactly where the pen was on the screen at any given moment. Once that was determined, the computer could then draw a cursor at that location. Also in 1961 another student at MIT, Steve Russell, created the first video game, E. E. Zajac, a scientist at Bell Telephone Laboratory (BTL), created a film called "Simulation of a two-giro gravity attitude control system" in 1963. 
During 1970s, the first major advance in 3D computer graphics was created at UU by these early pioneers, the hidden-surface algorithm. In order to draw a representation of a 3D object on the screen, the computer must determine which surfaces are "behind" the object from the viewer's perspective, and thus should be "hidden" when the computer creates (or renders) the image. 
In the 1980s, artists and graphic designers began to see the personal computer, particularly the Commodore Amiga and Macintosh, as a serious design tool, one that could save time and draw more accurately than other methods. In the late 1980s, SGI computers were used to create some of the first fully computer-generated short films at Pixar. The Macintosh remains a highly popular tool for computer graphics among graphic design studios and businesses. Modern computers, dating from the 1980s often use graphical user interfaces (GUI) to present data and information with symbols, icons and pictures, rather than text. Graphics are one of the five key elements of multimedia technology. 
3D graphics became more popular in the 1990s in gaming, multimedia and animation. In 1996, Quake, one of the first fully 3D games, was released. In 1995, Toy Story, the first full-length computer-generated animation film, was released in cinemas worldwide. Since then, computer graphics have only become more detailed and realistic, due to more powerful graphics hardware and 3D modelling software.

\thispagestyle{fancy}
\pagebreak

\thispagestyle{fancy}

\section{Characteristics of DBMS }
A number of characteristics distinguish the database approach from the much older approach of programming with files. In traditional file processing, each user defines and implements the files needed for a specific software application as part of programming the application. For example, one user, the grade reporting office, may keep files on students and their grades. Programs to print a student’s transcript and to enter new grades are implemented as part of the application. A second user, the accounting office, may keep track of students’ fees and their payments. Although both users are interested in data about students, each user maintains separate files— and programs to manipulate these files—because each requires some data not avail able from the other user’s files. This redundancy in defining and storing data results in wasted storage space and in redundant efforts to maintain common up-to-date data. In the database approach, a single repository maintains data that is defined once and then accessed by various users. In file systems, each application is free to name data elements independently. In contrast, in a database, the names or labels of data are defined once, and used repeatedly by queries, transactions, and applications. The main characteristics of the database approach versus the file-processing approach are the following:\\
\begin{itemize}
\item \textbf{} Self-describing nature of a database system.
\item \textbf{} Insulation between programs and data, and data abstraction. 
\item \textbf{} Support of multiple views of the data
\item \textbf{} Sharing of data and multiuser transaction processing.
\end{itemize}

\subsection{Self-Describing Nature of a Database System}
A fundamental characteristic of the database approach is that the database system contains not only the database itself but also a complete definition or description of the database structure and constraints. This definition is stored in the DBMS catalog, which contains information such as the structure of each file, the type and storage format of each data item, and various constraints on the data. The information stored in the catalog is called meta-data, and it describes the structure of the primary database. The catalog is used by the DBMS software and also by database users who need information about the database structure. A general-purpose DBMS software package is not written for a specific database application. Therefore, it must refer to the catalog to know the structure of the files in a specific database, such as the type and format of data it will access. The DBMS software must work equally well with any number of database applications—for example, a university database, a banking database, or a company database—as long as the database definition is stored in the catalog. In traditional file processing, data definition is typically part of the application programs themselves. Hence, these programs are constrained to work with only one specific database, whose structure is declared in the application programs. For example, an application program written in C++ may have structure or class declarations, and a COBOL program has data division statements to define its files. Whereas file-processing software can access only specific databases, DBMS software can access diverse databases by extracting the database definitions from the catalog and using these definitions.
\thispagestyle{fancy}

\subsection{Insulation between Programs and Data, and Data Abstraction}
In traditional file processing, the structure of data files is embedded in the application programs, so any changes to the structure of a file may require changing all programs that access that file. By contrast, DBMS access programs do not require such changes in most cases. The structure of data files is stored in the DBMS catalog separately from the access programs. We call this property program-data independence. In some types of database systems, such as object-oriented and object-relational systems users can define operations on data as part of the database definitions. An operation (also called a function or method) is specified in two parts. The interface (or signature) of an operation includes the operation name and the data types of its arguments (or parameters).The implementation (or method) of the operation is specified separately and can be changed without affecting the interface. User application programs can operate on the data by invoking these operations through their names and arguments, regardless of how the operations are implemented. This may be termed program-operation independence. The characteristic that allows program-data independence and program-operation independence is called data abstraction. A DBMS provides users with a conceptual representation of data that does not include many of the details of how the data is stored or how the operations are implemented. Informally, a data model is a type of data abstraction that is used to provide this conceptual representation. The data model uses logical concepts, such as objects, their properties, and their interrelationships, that may be easier for most users to understand than computer storage concepts. Hence, the data model hides storage and implementation details that are not of interest to most database users.
\thispagestyle{fancy}

\subsection{Support of Multiple Views of the Data}
A database typically has many users, each of whom may require a different perspective or view of the database. A view may be a subset of the database or it may contain virtual data that is derived from the database files but is not explicitly stored. Some users may not need to be aware of whether the data they refer to is stored or derived. A multiuser DBMS whose users have a variety of distinct applications must provide facilities for defining multiple views.
\thispagestyle{fancy}

\subsection{Sharing of Data and Multiuser Transaction Processing}
A multiuser DBMS, as its name implies, must allow multiple users to access the database at the same time. This is essential if data for multiple applications is to be integrated and maintained in a single database. The DBMS must include concurrency control software to ensure that several users trying to update the same data do so in a controlled manner so that the result of the updates is correct. For example, when several reservation agents try to assign a seat on an airline flight, the DBMS should ensure that each seat can be accessed by only one agent at a time for assignment to a passenger. These types of applications are generally called online transaction processing (OLTP) applications. A fundamental role of multiuser DBMS software is to ensure that concurrent transactions operate correctly and efficiently. The concept of a transaction has become central to many database applications. A transaction is an executing program or process that includes one or more database accesses, such as reading or updating of database records. Each transaction is supposed to execute a logically correct database access if executed in its entirety without interference from other transactions. The DBMS must enforce several transaction properties. The isolation property ensures that each transaction appears to execute in isolation from other transactions, even though hundreds of transactions may be executing concurrently. The atomicity property ensures that either all the database operations in a transaction are executed or none are.

\thispagestyle{fancy}




\section{Applications of DBMS}
Applications where we use Database Management Systems are: \\
\begin{itemize}
\item \textbf{Telecom:} There is a database to keeps track of the information regarding calls made, network usage, customer details etc. Without the database systems it is hard to
maintain that huge amount of data that keeps updating every millisecond.
\item \textbf{Industry:} Where it is a manufacturing unit, warehouse or distribution centre, each one needs a database to keep the records of ins and outs. For example distribution
centre should keep a track of the product units that supplied into the centre as well as
the products that got delivered out from the distribution centre on each day; this is
where DBMS comes into picture.
\item \textbf{Banking System: } For storing customer info, tracking day to day credit and debit
transactions, generating bank statements etc. All this work has been done with the help
of Database management systems.
\item \textbf{Education Sector: } Database systems are frequently used in schools and colleges to store and retrieve the data regarding student details, staff details, course details, exam
details, payroll data, attendance details, fees details etc. There is a hell lot amount of
inter-related data that needs to be stored and retrieved in an efficient manner.
\item \textbf{Online Shopping: }You must be aware of the online shopping websites such as
Amazon, Flip kart etc. These sites store the product information, your addresses and
preferences, credit details and provide you the relevant list of products based on your
query. All this involves a Database management system.
\end{itemize}
\thispagestyle{fancy}
\newpage
\thispagestyle{fancy} 