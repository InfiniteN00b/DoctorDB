\chapter{Testing}
There are several ways to test a doctor appointment booking database system, including:

\begin{itemize}
    \item \textbf{Functional testing:} This type of testing verifies that the system performs all of its intended functions correctly. This can include testing the system's ability to create, read, update, and delete appointments, as well as testing for any errors that may occur during these processes.
    
    \item \textbf{User acceptance testing:} This type of testing involves testing the system with real users to ensure that it is user-friendly and easy to use. This can include testing the system's navigation, search functionality, and appointment scheduling process.
    
    \item \textbf{Performance testing:} This type of testing evaluates how well the system performs under different workloads and conditions. This can include testing the system's response time, scalability, and ability to handle a large number of concurrent users.

    \item \textbf{Security testing:} This type of testing evaluates the system's security measures and checks if it is protected against unauthorized access, data breaches, and other security threats.

    \item \textbf{Integration testing:} This type of testing verifies that the system integrates correctly with other systems and applications.

    \item \textbf{Regression testing:} This type of testing is used to confirm that changes to the system have not introduced new bugs or broken existing functionality.

    \item \textbf{Unit testing:} This type of testing is used to check the individual units of code that make up the system.
\end{itemize}








