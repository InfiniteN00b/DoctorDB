%============================= abstract.tex================================
\chapter*{Abstract}%
%\addcontentsline{toc}{chapter}{\numberline{}Abstract}%
\addcontentsline{toc}{chapter}{Abstract}%

The Doctor Appointment Booking System is a web-based application that allows patients to book appointments with medical professionals online. The system allows patients to view the availability of doctors, schedule appointments, and receive confirmation of their appointments via email or text message. The system also includes a calendar view for doctors to manage their appointments and a dashboard for the admin to monitor and manage the appointments. The system aims to improve the efficiency and convenience of the appointment booking process for patients and medical professionals. Additionally, the system can be integrated with electronic medical record systems to easily access patient information. Overall, the Doctor Appointment Booking System is designed to streamline the appointment booking process and improve patient care.

1. MySQL: This is a relational database management system that can be used to store and manage the data for the project, such as patient information, doctor availability, and appointment schedules.

2. React: This is a JavaScript library for building user interfaces. It can be used to create a dynamic and interactive front-end for the system, allowing patients to view doctor availability and schedule appointments.

3. Node.js: This is a JavaScript runtime that can be used to build the back-end of the application. It can be used to create the server-side logic for the system, such as handling requests and managing the database.

4. Express.js: This is a web framework for Node.js that can be used to simplify the development of the back-end by providing a set of tools for handling routing, middleware, and other server-side tasks.

5. Axios: This is a JavaScript library that can be used to make HTTP requests from the front-end to the back-end.

6. Material-UI: This is a popular UI library for React that can be used to provide pre-built UI components that follow the Material Design guidelines.

7. ESlint, Prettier and Husky: These are tools that can be used to ensure code quality and maintainability, by running linting and formatting checks, and preventing commits with errors.

8. Webpack: This is a module bundler that can be used to bundle and optimize the front-end code for production.

The system is built using React as the front-end framework, and Node.js, Express.js and MySQL as the back-end stack. Passport.js is used for handling user authentication and authorization, Webpack for bundling the front-end code for production and Babel for transpiling the code to ensure compatibility with older browsers. Axios is used for making HTTP requests to the back-end. Material-UI is used to provide pre-built UI components that follow the Material Design guidelines. ESlint, Prettier and Husky are used to ensure code quality and maintainability.

The system aims to improve the efficiency and convenience of the appointment booking process for both patients and medical professionals. Additionally, the system can be integrated with electronic medical record systems for easy access to patient information. Overall, the Doctor Appointment Booking System application is designed to streamline the appointment booking process and improve patient care.



\thispagestyle{plain}
%=======================================================================

