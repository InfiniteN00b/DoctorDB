\chapter{Implementation}
\begin{itemize}
    \item 
    \text{To connect MySQL and React i.e FrontEnd and BackEnd connectivity}
\begin{verbatim}
let connection = mysql.createConnection({
    host: 'localhost',
    user: 'root',
    password: 'root',
    database: 'doctor',
    port: 3306,
    multipleStatements: true,
    
});
connection.connect(function(err) {
    if (err) {
        console.error('error connecting: ' + err.stack);
        process.exit(-1);
    }
    console.log('connected as id ' + connection.threadId);
});
\end{verbatim}
    \item
    \text{Setting Access controls}
    \begin{verbatim}
    app.use(function(req, res, next) {
        console.log(req.session);
        console.log('!OPTIONS');
        res.header('Access-Control-Allow-Origin', req.headers.origin);
        res.header('Access-Control-Allow-Credentials', "true");
        res.header('Access-Control-Allow-Methods', 'GET,PUT,POST,DELETE');
        next();
    });
    \end{verbatim}
    \item
    \text{Login API: }
    \begin{verbatim}
    const handleSubmit = (event) => {
        event.preventDefault();
        const data = new FormData(event.currentTarget);
        fetch('/api/login', {
            method: 'POST',
            mode: 'cors',
            credentials: 'include',
            headers: { 
                'Content-Type': 'application/json'
            },
            body: JSON.stringify({
                email: data.get('email'),
                password: data.get('password')
            })
        })
        .then(response => response.json())
        .then(data => {
            console.log('Success:', data);
            window.location = '/'
        }
        )
        .catch((error) => {
            console.error('Error:', error);
        });
    };
    \end{verbatim}
    \item
    \text{SignUp API: }
    \begin{verbatim}
    const handleSubmit = (event) => {
        event.preventDefault();
        const data = new FormData(event.currentTarget);
        fetch('/api/signup', {
            method: 'POST',
            mode: 'cors',
            credentials: 'include',
            headers: {
                'Content-Type': 'application/json'
            },
            body: JSON.stringify({
                firstName: data.get('firstName'),
                lastName: data.get('lastName'),
                email: data.get('email'),
                password: data.get('password')
            })
        })
        .then(response => response.json())
        .then(data => {
            console.log('Success:', data);
            window.location.reload()
        })
        .catch((error) => {
            console.error('Error:', error);
        });
    };
    \end{verbatim}
    \item
    \text{Booking Appointment from Patient: }
    \begin{verbatim}
  function submitAppointment() {
    fetch('/api/appointment', {
        method: 'POST',
        headers: {
            'Content-Type': 'application/json',
        },
        body: JSON.stringify({
            "pid": user.id,
            "did": doctor.id,
            "timeSlot": 
                value.toDate().toISOString().slice(0, 19).replace('T', ' '),
        }),
    })
    .then(response => response.json())
    .then(data => {
        console.log('Success:', data);
        SnackbarUtils.success("Appointment Added Successfully")
        setOpen(false);
    })
    .catch((error) => {
      SnackbarUtils.error("Error Adding Appointment")
      console.error('Error:', error);
    });
  }
    \end{verbatim}
    \item
    \text{Doctor Accepting Appointments: }
    \text{FrontEnd fetcg API: }
    \begin{verbatim}
  const handleClick = (id) => {
    fetch("/api/accepted/" + id)
      .then((res) => {
        if (res.status !== 200) {
          SnackbarUtils.error("Error accepting appointments");
          throw new Error("Error accepting appointments");
        }
        return res.json();
      })
      .then((data) => {
        setAppointments(data);
        window.location.reload();
      })
      .catch((err) => {
        console.error(err);
        SnackbarUtils.error("Error accepting appointments");
      })
  }
    \end{verbatim}
    \item
    \text{post request to BackEnd for altering appointment table:}
    \begin{verbatim}
app.get('/api/accepted/:bookingID', function(req, res) {
    if (!req.session.user) {
        res.status(401).json({message: 'Unauthorized'});
        return;
    }
    if (!(req.session.user.UserRoleid && req.session.user.UserRoleid == '1')) {
        console.log( "doctor", req.session.user);
        res.status(400).json({message: ''});
        return;
    }
    const query = `Update Appointment SET accepted = true WHERE book_id = ${req.params.bookingID};`;
    connection
        .query(query, function(err, result) {
            if (err) {
                console.log(err);
                res.status(500).json(err);
            } else {
                res.status(200).json(result);
            }
        });
});
    \end{verbatim}
\end{itemize}



%Paste your code(limited to 5 pages)